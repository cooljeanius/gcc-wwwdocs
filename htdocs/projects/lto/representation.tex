Link-time optimization requires that the compiler emit some
information during initial compilation.  This information will be read
back in by the link-time optimizer.  This section discusses the
representation of that information.

\begin{requirement}
  \label{req:interop}
  The representation used must be ``compiler-independent'' in the
  sense that it should be feasible for tools other than GCC (and
  written in languages other than C) to read the representation
  generated.
\end{requirement}

\begin{rationale}
  The simplest way to represent the information GCC needs would be to
  serialize the TREE data structures to disk.  Since these are the
  data structures GCC uses for optimization and code generation, we
  know that these data structures contain the information required.
  However, there are a number of disadvantages to this approach.

  First, different versions of the compiler will almost certainly be
  unable to interoperate.  Minor differences in the tree structure
  from version-to-version of the compiler are inevitable.  In fact,
  different builds of the same version of the compiler may not have
  precisely compatible tree structures; for example, for some
  host/target combinations, either 32-bit or 64-bit
  \code{HOST\_WIDE\_INT}s may be used, but the resulting in-memory tree
  representations are different.  It is highly desirable to support
  link-time optimization of relocatable files built by one provider on
  the systems of another provider, and there is no guarantee that both
  environments are necessarily using the same version of the compiler.

  Second, there are a large number of additional tools that would
  benefit from being able to access the information possessed by the
  compiler in the middle of its processing.  Lint-like source analysis
  tools, IDEs, and other similar tools are all potential clients.
  Various projects to generate XML from GCC have been attempted in
  order to try to facilitate this kind of usage.
\end{rationale}

It might be possible to use a well-known representation format, such
as Java bytecode or Microsoft's CIL.  However, in addition to concerns
about possible intellectual property issues and alignment with
particular vendors, we also concluded that neither of these formats
was well-suited to GCC's needs.  Java bytecode is too Java-specific
for a multi-language compiler.  CIL is too closely tied to Microsoft's
virtual machine architecture.  Therefore, we concluded that it is
necessary to develop a new representation format.

\begin{requirement}
  The representation format should be extensible to contexts other
  than link-time optimization.
\end{requirement}

\begin{rationale}
  The primary purpose of the representation format is for link-time
  optimization.  However, while the link-time optimizer may want to
  process a language-neutral representation of the program that has
  been partially optimized before emission, other tools (such as the
  IDEs and source-analysis tools mentioned above) will likely want to
  access an unoptimized representation of the program that contains
  more information about the way in which the user originally
  formulated the program.  In particular, an implementation of the C++
  ``export'' keyword will want a representation of the program that is
  C++-specific.  GCC's TREE representation is sufficiently flexible to
  represent these various levels; the on-disk representation should be
  similarly flexible.
\end{rationale}

\begin{requirement}
  The semantics of the representation (for each supported level of
  representation) must be well-specified.
\end{requirement}

\begin{rationale}
  In order to facilitate the interoperability described in
  Requirement~\ref{req:interop}, we must have a well-specified
  explanation of the meaning of the representation.  Ideally, we would
  have an operational semantics capable of formally describing
  execution of the input program, but that goal is not realistically
  achievable; instead, we will have to settle for an informal document
  describing the semantics.

  Each level of representation should be documented, as part of the
  GCC manual, and handled through the normal GCC development process.
  We do not intend to encourage ad-hoc extensions to the format for
  the use of particular individual needs.
\end{rationale}

\begin{requirement}
  The representation used for link-time optimization should correspond
  approximately to the GIMPLE TREE representation.
\end{requirement}

\begin{rationale}
  The RTL level would be too low-level a representation for
  link-time optimization as its use would preclude the use of the
  increasingly powerful TREE optimizers.

  GIMPLE is a language-independent representation that has been
  simplified to make it amenable to optimization.  However, high-level
  information, such as the types of expressions remains.

  Before emitting information for link-time optimization, it may be
  useful to apply existing optimizations that reduce program size,
  such as constant propagation, dead code elimination, and, perhaps, a
  limited form of inlining.  Functions with internal linkage that are
  never referenced can be eliminated.

  We do not yet have an opinion as to whether the representation
  should be in SSA form.  While the information provided by SSA form
  is quite useful, it can be easily regenerated.  PHI nodes, and the
  use of additional variables, may so bloat the representation that it
  is more efficient to emit a non-SSA form.

  The use of general compression algorithms may be undesirable, to the
  extent that they make decoding the data difficult, including
  preventing random access to the data.  However, it might still be
  desirable to use such an algorithm, if the compression achieved is
  sufficiently significant.  In addition, some forms of
  context-sensitive compression may be useful.  Almost all
  externalized byte code representations use a representation that is
  a variant of three-address code.  These representations often
  require a large number of intermediate variables.  Schemes, such as
  dividing the program into blocks such that a limited number of
  variables are present in each block, and then using small integers
  to refer to the registers with a given block, can reduce the cost.
\end{rationale}

\begin{requirement}
  The representation format must be designed to support forward
  evolution.  For example, the format should permit the addition of
  new operators as such operators are added to GIMPLE.  It should be
  possible to perform link-time optimization using input object files
  generated with different versions of the compiler, provided that the
  link-time optimizer used is that corresponding to the newest
  compiler used to build the object files.
\end{requirement}

\begin{note}
  This requirement is not meant to imply that an old version of the
  link-time optimizer should be able to process objects created by a
  newer version of the compiler.  Rather, a new version of the
  link-time optimizer should be able to process objects created by an
  older version of the compiler.  There may, occasionally, be cases
  where that is not possible, and the representation format should
  contain version numbers that permit the link-time optimizer to
  identify such cases.
\end{note}

\begin{rationale}
  Since the format is intended for use not only by the link-time
  optimizer, but also by other tools, the format must be fundamentally
  stable.  However, as new optimizations and new language features are
  added to GCC, it may occasionally be necessary to add new operations
  to GIMPLE, new kinds of types, and so forth.  Therefore, the format
  must support forward evolution.
\end{rationale}

\subsection{Declarative Representation}

This subsection outlines the representation of declarative entities,
i.e., types, variables, functions (their declarations, not their
bodies), and other similar entities.  The bodies of functions are
discussed in Section~\ref{sec:execrep} below.

The representation for types will be DWARF-3.  Although the DWARF-3
standard was originally designed for providing debugging information,
it's representation for types provides nearly all of the information
required for link-time optimization, including the location of
non-static data members and base classes.  Furthermore, DWARF-3
specifically provides for vendor extension, allowing us to record any
additional information that we might require.  In particular, we will
extend DWARF-3 to represent relevant GNU attributes.  

\begin{note}
Description of GNU attributes in DWARF-3 will also be of use to GDB
and other debuggers, to the extent that, for example, these attributes
affect the calling conventions of functions using these types.  While
this proposal contemplates the use of the DWARF-3 extensions in
service of link-time optimization, there is no reason the same
extensions should not be used as part of the conventional debugging
information.
\end{note}

DWARF-3 has several other advantages for our purposes, including:
\begin{itemize}
\item DWARF-3 is a well-specified standard.

\item GCC already knows how to emit DWARF-3 information.  Therefore,
  the effort required to emit the information will be much smaller.

\item Various tools, including \code{readelf}, know how to interpret
  and display DWARF-3 information.  Therefore, it will be possible to
  more easily verify the generated information, and construction of the
  portion of the link-time optimizer that will read this information
  and reconstitute appropriate TREE nodes will be easier.
\end{itemize}

The representation for declarations will also be DWARF-3.  For
function declarations and for variables with static storage duration
(i.e., ``global'' variables) this representation is very natural.  For
automatic variables, the representation is less natural, in that
DWARF-3 scope information refers to byte offsets within code sections.
In other words, the DWARF-3 representation for a local variable says
that it comes into scope at offset $N_0$ (from the start of a
designated section) and goes out of scope at offset $N_1$.  References
into the ordinary code sections clearly will not be useful to the
link-time optimizer.  Therefore, a separate DWARF-3 section will be
used which references not the usual code section, but, rather, the
section used by the executable representation.

Some variables with static storage duration require initialization.
If the initialization should be performed dynamically, no special
handling is required; the dynamic initialization will appear in the
ordinary code stream.  If the initialization is performed statically,
the initializer will be recovered directly from the data section in
the relocatable file, as there is no convenient representation for
initialization in DWARF-3.

\subsection{Executable Representation}
\label{sec:execrep}

The executable portion of the intermediate code will be ``bytecode''
for a stack machine hereinafter known as the GNU Virtual Machine
(GVM).  The stack machine will represent code at approximately the
same level represented by the GIMPLE level of TREE.  The use of a
stack removes any need for pickled pointers to represent expression
trees.  The TREE representation of the executable code can be readily
reconstituted from the stack code.

The GVM will share some features with Sun's Java Virtual Machine (JVM)
and Microsoft's Common Intermediate Language (CIL).  However, there
will also be some important differences:
\begin{description}

\item[portability] Like JVM bytecode, the format of the GVM bytecode
  will be architecture-independent.  However, unlike JVM bytecode, it
  will not be possible to reuse the GVM bytecode from one platform on
  another; for example, the bytecode generated by an IA32 GNU/Linux
  will not be reusable as part of a PowerPC AIX program.
  
\item[executability] Given that the GVM code will be translated into
  executable code, it is highly likely that this code could be
  directly executed.  However, no particular attention will be given
  to make this easy.  What is important is to be able to transport the
  intermediate code efficiently from the compiler to the link time
  optimizer.

\item[verification] Some bytecode formats permit verification; i.e., a
  tool can check that executing the bytecode will not result in
  certain kinds of unsafe behaviors.  The Java bytecode format is
  fully verifiable.  The CIL format provides optional verification.
  The GVM format will not be designed to support verification because
  most of our target languages explicitly permit operations that would
  be considered unsafe by a verifier.

\item[operators] The operators will be drawn from two sets: those
  that manipulate the stack ({\code load, store, duplicate, swap)} and
  the operators used in GIMPLE.

  \begin{note} 
  We believe that it may be useful to add some higher-level operations
  to GIMPLE, such as array operations, virtual functions calls,
  dynamic casts, parallel execution blocks, and high level loops.  To
  the extent that these higher-level operators will facilitate
  additional optimizations, they should be present in GIMPLE, and,
  therefore, in the stack machine.  As GIMPLE evolves, the stack
  machine should evolve in parallel.
  \end{note}

\item[the stack] The stack is an unbounded array of expressions.  It
  will be an invariant of the code generated that the stack will be
  empty at basic block boundaries.  Basic block boundaries will be
  explicitly indicated in the stack machine code.

  JVM and CIL codes utilize a register allocation pass to generate
  their intermediate code. This register allocation obscures the code
  in a way that makes debugging more difficult and requires
  significant resources on both the input and output sides.  Therefore,
  the generation of GVM bytecode will not require register allocation.

\item[blocks] There will be several types of explicit blocks in the
  code stream. These are basic blocks, bind expressions, parallel
  execution blocks.  

\item[the registers] The registers of the GVM correspond one for one
  with the local variables at the GIMPLE level.  Each local variable
  will be assigned an index, and a symbol table will be provided to
  keep the type of each of these temps.  Types will be given as 
  references to the corresponding DWARF-3 DIEs representing those
  types.

\item[types] The stack is dynamically typed.  The type of an element on the
  stack is determined from the operation used to set that item.  The
  result type of a generic operation (e.g., addition) is determined
  from the inputs to that operation.  Some operations (e.g., casts)
  will explicitly indicate the result of the operation.

\end{description}

