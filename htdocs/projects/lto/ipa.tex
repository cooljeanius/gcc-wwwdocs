\documentclass{article}

\newenvironment{note}{\emph{Note:}}{\emph{-- End note.}}

\newenvironment{rationale}{\emph{Rationale:}}{\emph{-- End rationale.}}

\newtheorem{requirement}{Requirement}

\newcommand{\code}[1]{\texttt{#1}}

\newcommand{\toss}[1]{}

\newcommand{\fixme}[1]{\textit{FIXME: #1}}

\begin{document}
\title{Link-Time Optimization in GCC: Requirements and High-Level Design}

\maketitle

\thispagestyle{empty}

\section{Introduction}
Popular programming languages, such as C, C++, and Fortran, use
``separate compilation'' to facilitate building large programs.
The program is written in various parts, which are usually stored as
separate files in the filesystem.  Each part is compiled in isolation.
The linker combines the separately compiled parts into the executable
image.  

Although separate compilation is unarguably a useful technique, it has
the disadvantage that the compiler is unable to perform optimizations
that rely on knowledge about more than one part of the program.
Therefore, many modern compilers perform additional optimizations when
the program (or a portion of the program) is linked.  At link-time,
these compilers perform optimizations that would be impossible given
any single file.  For example, functions can be inlined across files,
and optimizations like dead-code elimination, constant-folding, and
global data allocation can be performed on all of the program parts
simultaneously.  Most of the optimizations performed are already
performed when compiling a single file; therefore, it is in general
possible to reuse much of the infrastructure already present in the
compiler.

At present, GCC contains a simple form of this functionality which
relies on compiling multiple source files simultaneously.  This
mechanism is unsatisfactory if the various source files that make up a
program are not all available at once (as is true in many large
programs), are written in different languages, or need to be compiled
with different compiler options.  Furthermore, the current
functionality is provided only for C programs, and will not be easy to
extend to C++ or Fortran programs.

We believe that in order to provide link-time optimization competitive
with that provided by other compilers, we must adopt the same
technique that they use.  In particular, the compiler must save data
about each program part during compilation, and at link-time, this
data must be read back into the compiler so that the compiler can
perform optimizations across multiple program parts.

This document is both an informal requirements document and an
informal design document.  It provides high-level requirements that we
think must be met by any implementation of link-time optimization.
This document also provides a sketch of the implementation approach we
intend to use.

The next section explains the semantics of the link-time optimizer,
i.e., the way in which it must behave.  Section~\ref{Architecture}
provides a high-level description of the modifications required to the
toolchain to support the link-time optimizer.
Section~\ref{Representation} sketches the on-disk representation of
program parts that will be emitted during compilation, and retrieved
during link-time optimization.

\section{Semantics}\label{Semantics}
This section explains the semantics that must be observed by the
link-time optimizer.  The first subsection discusses the external
interface for the link-time optimizer, including its relationship to
the compiler driver.  The second subsection explains how the link-time
optimizer will combine multiple translation units into a single unit.

\subsection{External Interface}

\begin{requirement}
 Use of link-time optimization must require only the addition of a
 single command-line option at compile-time and link-time.  The
 link-time optimizer must not require access to input source files, or
 any other data other than that embedded in the object files by the
 compiler.
\end{requirement}

\begin{note}
  The compiler could choose to embed the entire source file in the
  object file without violating this requirement.  The point of this
  requirement is not to restrict the kind of information which the
  compiler embeds in the relocatable object file; rather, the point is
  that the inputs to the link time optimizer should be the same
  relocatable object files the user would have at hand if link-time
  optimization were not in use.
\end{note}

\begin{rationale}
  The objective of this requirement is ease of use.  Users should not
  have to maintain ``on-the-side'' databases, or otherwise 
  modify the workflow to which they are accustomed.  When link-time
  optimization is in use, the compiler should still generate assembly
  files, the assembler should still generate object files, and the
  linker should still combine those object files.  Any other paradigm
  will require overly invasive changes on the part of users, which
  will be a significant barrier to adoption.  The paradigm suggested
  here will require only the addition of a single option to the 
  \code{CFLAGS} make variable to enable link-time optimization in many
  programs.
\end{rationale}

\begin{requirement}
  The link time optimizer must be able to operate on a subset of the
  complete program.  The input to the link-time optimization step will
  be one or more relocatable files (e.g., ELF \code{ET\_REL} files) .
  The output from this step will be a single relocatable file
  containing the (optimized) contents of the input files.
\end{requirement}

\begin{rationale}
  Whole-program optimization is the specific case of link-time
  optimization in which it is assumed that the optimizer is able to
  see the entire program.  That assumption is useful in that, for
  example, functions that are unreferenced can be discarded, even if
  they have external linkage.  However, limiting the link-time
  optimizer to the whole-program case would preclude link-time
  optimization in common cases.  For example, the use of a single
  hand-written assembly code module might prevent link-time
  optimization.  Or, it might be impossible to perform link-time
  optimization when creating a shared library.

  Therefore, while a whole-program mode would indeed be useful, 
  it is essential that the link-time optimizer be able to operate on
  a subset of the object files that will make up the eventual
  program.
\end{rationale}

\begin{requirement}
 \label{req:binding}
 The set of symbols with global or weak binding (in ELF, any symbol with an
 \code{ELFXX\_ST\_BIND} value of \code{STB\_GLOBAL} or
 \code{STB\_WEAK}) defined in the output relocatable file must be the
 union of such symbols in the input relocatable files.
\end{requirement}

\begin{requirement}
 \label{req:debug}
 Those entities with non-global, non-weak binding in the input files
 which are present in the output file must have the same names in the
 output file that they have in the input file.
\end{requirement}

\begin{rationale}
  It is valid to combine two C translation units which both define a
  function with internal linkage named \code{f}.  In order to support
  debugging of optimized programs, these functions must continue to be
  named \code{f} in the output file.
\end{rationale}

\begin{requirement}
 \label{req:optact}
 The link-time optimizer must take one of the following actions:
 \begin{itemize}
 \item Combine the input relocatable files into an output relocatable
   file.
 \item Issue an error message indicating that the combination was
   invalid, i.e., that any program containing this combination would be
   invalid, according to the relevant language standards.
 \item Refuse to the perform the combination, despite the fact that the
   combination is valid.
 \end{itemize}
\end{requirement}

\begin{note}
 A trivial implementation of Requirement~\ref{req:binding} would be to
 use the traditional linker, invoked with the \code{-r} option, as the
 link-time ``optimizer''.  This ``optimizer'' would only make use of
 the first two options above.  The intent of the requirement is that
 the driver, when invoked to perform link-time optimization, will
 behave as a ``smart'' drop-in replacement for \code{ld~-r}.
\end{note}

\begin{requirement}
 \label{req:optcomb}
 If the link-time optimizer refuses to perform a valid combination,
 the compiler driver must run \code{ld~-r} (or an appropriate
 system equivalent) to combine the object files.
\end{requirement}

\begin{rationale}
 Certain combinations of object files may be permitted by the language
 standards, but may require undue effort to implement in the link-time
 optimizer.  In addition, the level of support (if any) for link-time
 optimization may vary from architecture to architecture or from
 system to system.  In these cases, rather than issue an error, which
 would disrupt the build compilation process, the compiler driver will
 invoke \code{ld~-r} (so that the build process proceeds.  This
 mechanism will make it possible for users to incorporate use of the
 link-time optimizer in their build processes without needing to
 conditionalize that use.
\end{rationale}

\begin{requirement}
  If command-line options used to compile the input relocatable object
  files are incompatible, the link-time optimizer must refuse to
  perform the combination.
\end{requirement}

\begin{rationale}
  The use of GCC command-line options (such as \code{-ftrapv}) are
  global options that affect the semantics of GIMPLE.  Ideally, the
  impact of these options might be represented directly in GIMPLE; for
  example, we might have both ``trapping addition'' and ``non-trapping
  addition'' operators, rather than using a global variable to
  indicate what kind of addition is performed by a single addition
  operator.  However, at present, GCC's internal representation cannot
  simultaneously describe both operations.  Therefore, if one input
  object file uses command-line options incompatible with the options
  used to compile another, the link-time optimizer must reject the
  combination.
\end{rationale}

\subsection{Combination of Translation Units}

This document describes the semantics for combinations of translation
units, assuming that all of the input translation units were written
in either the C or C++ programming languages.  This simplifying
assumption is not meant to indicate that the mechanisms used are
intended to be language-specific; rather, it is explicitly our intent
that the mechanisms be language-independent, and that it be possible
to add support for new programming languages to the existing link-time
optimization framework.  

However, each additional language will require some amount of thought
about what it means to combine code written in that language with the
code written in other languages.  The key questions that must be
answered for each language are:
\begin{itemize}
\item Which entities are the same?  For example, is type $T_1$ written
  in language $L_1$ the same type as type $T_2$ written in language
  $L_2$?
\item If two entities or two types are the same, is the combination
  valid?  For example, two variables with external linkage and the
  same name in the relocatable object file \emph{must} be the same
  variable, because any ordinary linker would consider them the same.
  However, if the variables do not have the same type, then the
  combination is invalid.
\end{itemize}
The remainder of this section endeavors to answer those questions for
the C and C++ programming languages.

The C and C++ programming language standards are written in terms of
individual ``translation units'' which are then combined.  In GCC,
each relocatable file is the compiled form a single translation
unit.  The language standards state that certain combinations of
translation units (in order to form a single program) are valid, while
others are invalid.  This section explains describes the requirements
that the link-time optimizer must obey when combining translation
units.

For example, it is not valid in C to combine two translation units if
both declare a variable with external linkage named \code{a}, but in
one translation unit the variable is given type \code{int} and in the
other translation unit the variable is given type \code{double}.  It
is also invalid to combine two translation units that define a
function with external linkage named \code{f}.  Of these, only the
second invalid combination is diagnosed by most linkers.

\begin{note}
 As per Requirement~\ref{req:optcomb}, the link-time optimizer may
 choose not to combine translation units, even though such a
 combination would be valid.  The requirements in this section apply
 only if the link-time optimizer does in fact perform the
 combination.  If it does not perform the combination (and, therefore,
 relies upon the compiler driver to use the traditional linker to
 perform the combination), then, of course, these requirements do not
 apply.
\end{note}

\begin{requirement}
  Valid combinations of translation units must behave as required
  by the appropriate language standard(s).
\end{requirement}

\begin{requirement}
\label{req:invcomb}
  Invalid combinations of translation units which would result in a
  diagnostic if linked together by the traditional linker should also
  result in a diagnostic when combined via link-time optimization.
\end{requirement}

\begin{note}
A trivial implementation of Requirement~\ref{req:invcomb}
would be to directly invoke the linker, solely for the purpose of
obtaining any diagnostics, before performing link-time optimizations. 

These requirements do not preclude diagnosing invalid combinations
that would not be diagnosed in the ordinary process of linking.  In
fact, some programmers may desire these additional diagnostics and
welcome the stricter standards conformance implied by such
diagnostics.  However, experience suggests that many programs contain
technically invalid combinations that, in practice, do not result in
problems.  A common example is the declaration of a variable in one
translation unit as \code{int} while the same variable is declared as
\code{long} in another translation unit.  On an ILP32 system, these
types, while not technically compatible, are both declarations of a
32-bit integer type.  The link-time optimizer need not perform such
invalid combinations, but issuing a fatal error may not be the most
appropriate action in such a situation.
\end{note}

In order to combine the input translation units, the link-time
optimizer must recognize that certain entities in one translation unit
are ``the same'' as entities in another translation unit.  Entities
that are the same must be ``merged'' into a single entity.  There are
three kinds of entities that must potentially be merged: types,
variables, and functions.

Without loss of generality, we consider only two translation units;
for expository purposes, we consider the combination of multiple
translation units as successive combinations of pairs.

For each kind of entity, we first describe how to determine whether or
not two entities are the same.  Then, once the entities are determined
to be the same, we indicate whether or not the combination is ``valid'',
``invalid'', or ``difficult''.  An invalid combination is a combination
prohibited by the relevant language standards; such a combination
represents an error in the input program.  A difficult combination is
one which is permitted by the relevant language standards, but in
which the combination would be fundamentally difficult to optimize.

\begin{requirement}
 \label{req:combs}
 The link-time optimizer should perform valid combinations and should
 issue (possibly fatal) errors for invalid combinations.  If the
 link-time optimizer elects not to perform a difficult combination, it
 must issue a diagnostic explaining the source of confusion.
\end{requirement}

\subsubsection{Variables and Functions}

This section explains how to tell if two variables, or two functions,
are the same, and, if they are the same, whether the combination is
valid, invalid, or difficult.

Let $E_1$ be a variable or function from the first translation unit and
$E_2$ be a variable or function from the second translation unit.

If either $E_1$ or $E_2$ has a binding other than global or weak, then
$E_1$ and $E_2$ are not the same.

Otherwise, $E_1$ and $E_2$ are the same if (and only if) they have the
same name, in the form that such names appear in the relocatable file
symbol table.  In particular, for languages like C++, in which the
names of entities in the source program cannot be directly represented
in traditional relocatable file formats, the names used in the
relocatable file symbol table are ``mangled'' by most compilers,
including the compilers in the GNU Compiler Collection.  It is these
mangled names that are used to determine identity.

\begin{note}
 GCC provides mechanisms for overriding the name used for an entity in
 a relocatable file, so that, for example, a variable named \code{i}
 in the source code may be named \code{j} in the relocatable file.  In
 that case, it is \code{j} (not \code{i}) that is used to determine
 identity.
\end{note}

The combination is invalid if and of the following conditions hold:
\begin{itemize}
\item $E_1$ is a variable and $E_2$ is a function, or vice versa.

\item $E_1$ and $E_2$ do not have the same type, in the sense of
Section~\ref{sec:types}, except that if the type of one of $E_1$ or $E_2$
involves an incomplete array type, this may be replaced with a complete
array types in the type of the other entity, so long as the element
types are the same.

\item Both $E_1$ and $E_2$ are definitions, but (a) neither $E_1$ nor
  $E_2$ is weak, and (b) $E_1$ and $E_2$ are not both C++ inline
  functions.

\item $E_1$ and $E_2$ have incompatible GNU attributes.
\end{itemize}

The combination is difficult if:
\begin{itemize}
\item $E_1$ and $E_2$ are functions, and the type of one uses an
  unprototyped function type, while the other does not.
\end{itemize}

\begin{note}
The C99 programming language permits multiple definitions of inline
functions, even if those definitions are not identical.  A C99 inline
function must always have a canonical, non-inline definition.  The
semantics are that the address of such a function is always that of
the canonical definition.  Calls may use either the inline definition
provided in the same translation unit as the caller, or the canonical
definition, but not an inline definition in some other translation
unit.  The simplest way to handle this complexity is for the compiler
to omit the inline definitions from the intermediate representation,
even if it has used them for early inlining by that point.  If we
elect to emit the inline definitions as well, then combinations
involving at least one C99 inline definition should probably be
considered difficult.
\end{note} 

Otherwise, the combination is valid.

If either $E_1$ or $E_2$ is a strong definition, that definition is
used.  Otherwise, if exactly one of $E_1$ and $E_2$ are definitions,
then that definition is used.  Otherwise both $E_1$ and $E_2$ are
declarations, or both are weak definitions, or both are C++ inline
definitions; in that case, the link-time optimizer may use either
$E_1$ or $E_2$.

\subsubsection{Types}
\label{sec:types}

This section explains how to tell if two types are the same, and, if
they are the same, whether the combination is valid, invalid, or
difficult.  

\begin{note}
  Types, in contrast to variables and functions, are not combined by
  traditional linkers.  Therefore, types present a problem that is
  less well-understood than the corresponding problem for variables
  and functions.
\end{note} 

Types with differing cv-qualifiers are never the same.  In the absence
of GNU attributes, corresponding integral types, corresponding
floating point types, and the \code{void} type are the same in all
translation units.  Array types are the same type if their element
types are the same and if the number of elements in the array is the
same.  Function types are the same type if the return type and
arguments types are the same.  Pointer (reference) types are the same
if the pointed-to (referred-to) types are the same.  Pointers to
non-static members are the same if the pointed-to types and class
types are the same.  In all of these cases, if GNU attributes are
present, types which would be the same be it not for their GNU
attributes may in fact be different due to the use of GNU attributes.

\begin{note}
  The preceding rules imply that type-equivalence for these basic types
  is ``structural'', i.e., does not depend on the name of the
  types.  The reason for this rule is that C and C++ \code{typedef}
  names are irrelevant, for the purposes of comparing types.
\end{note}

If $T_1$ and $T_2$ have the same name and are both enumeration types,
then $T_1$ and $T_2$ are the same.

If $T_1$ and $T_2$ have the same name (i.e., in C++, would have the
same \code{std::type\_info} object) and both are class 
types\footnote{Class types include those types
  declared with the \code{class}, \code{struct}, and \code{union}
  keywords},  
then $T_1$ and $T_2$ are the same.

\begin{note}
  The preceding rules imply that type-equivalence for class and
  enumeration types is not structural; rather, it depends on the names
  of the types.  However, the validity rules below impose a structural
  consistency on the types that are the same.
\end{note}

Otherwise, the combination is invalid if both types were written in
C++ translation units, and at least one of the following conditions hold:
\begin{itemize}
\item $T_1$ and $T_2$ are both complete enumeration types and do not have
  the same minimum and maximum values.

\item $T_1$ and $T_2$ are both complete class types and do not have the same
  size and alignment.

\item $T_1$ and $T_2$ are both complete class types do not have non-static
  data members (including those implicitly generated by the compiler)
  of the same types at the same offsets, or do not have the same base
  class types at the same offsets, or do not have the same set of
  virtual functions, declared in the same order.

\item $T_1$ and $T_2$ have incompatible GNU attributes.
\end{itemize}

\begin{note}
If two enumeration types have the same minimum and maximum values,
they have the same underlying type.
\end{note}

The combination is difficult if at least one type was written in the C
programming language, but the combination would be invalid if both
types were written in C++.

If either $T_1$ or $T_2$ is a complete type, then the complete type is
used.  Otherwise, the link-time optimizer may select either type.

%%% Local Variables: 
%%% mode: latex
%%% TeX-master: t
%%% End: 


\section{Architecture}\label{Architecture}
The link-time optimizer will not be an entirely separate toolchain
component.  Instead, the link-time optimizer will reuse technology
from the existing in order to provide a shorter time-to-solution, to
reduce the effort required to port to new platforms, and to provide a
mechanism for sharing optimization capabilities between the ordinary
and link-time optimizers.

Of course, many components of the toolchain must be modified to
support link-time optimization.  The remainder of this section
describes, at a high level, the modifications that will be made to the
established toolchain architecture.

\subsection{Emission of Information}

The compiler will be modified to emit new special sections containing
the stack machine representation of the tree structure.  (For details
on the format of the representation, see Section~\ref{sec:execrep}.)
The compiler will also be modified to emit additional DWARF-3
sections, as necessary, to describe variables in function bodies.  The
stack code will be emitted using numeric literals so that no assembler
support will be required.

\begin{note}
 Systems that do not support special sections may store the
 information required using some other mechanism.
\end{note}

\subsection{Link-Time Optimizer Front End}

A new GCC front end will be provided to serve as the link-time
optimizer. The input language for this front-end will be relocatable
object files, not programming language source code.  The front end
will extract the relevant data from the relocatable object files,
build a TREE representation for the input program, and, then, use the
same optimizers and back-ends already present in GCC.

In order to accommodate Requirement~\ref{req:debug}, the back end will
be modified to permit the generation of multiple symbols with the same
name, if those symbols have internal linkage.  This modification will
be conditional on appropriate assembler support.

\subsection{Assembler}

The assembler must be modified to permit multiple symbols with the
same name, if at most one such symbol has external linkage.
Obviously, there must be some way of indicating to which of these
symbols a particular reference applies.  The simplest solution to this
problem is to continue to use unique names for the symbols throughout
the majority of the assembly file, but to provide a mapping table
indicating how symbols should be renamed before they are emitted in
the relocatable file.

\subsection{Linker}

The linker contains logic for determining which set of objects should
be extracted from archives (typically, files with the \code{.a}
extension) when performing a link.  The linker will be modified to
provide a mode in which this information is emitted, but no actual
linking is performed.  This facility will allow the link-time
optimizer to work only with relocatable files and avoid duplication of
code already present in the linker.

\subsection{Driver}

The driver will be modified to invoke the link-time optimizer front
end at link time, when an appropriate option is provided on the
command-line.  If the link-time optimizer successfully produces an
assembly file, the driver will then invoke the assembler, and
thereby generate a new object file.  If the link-time optimizer exits
with a fatal error, the driver will exit with a non-zero exit code.
Finally, if the link-time optimizer indicates that it is unable to
optimize the input files, the driver will invoke the linker, with the
\code{-r} option, to perform an unoptimized partial link.  In this
last case, the link-time optimizer will have already issued a
diagnostic as per Requirement~\ref{req:combs}.

The driver will also provide a mode where the partial link performed
by the link time optimizer is followed by a full link.  Using this
mode, it will be possible to take advantage of link-time optimization
simply by providing one additional option at link time.


\section{Representation}\label{Representation}
Link-time optimization requires that the compiler emit some
information during initial compilation.  This information will be read
back in by the link-time optimizer.  This section discusses the
representation of that information.

\begin{requirement}
  \label{req:interop}
  The representation used must be ``compiler-independent'' in the
  sense that it should be feasible for tools other than GCC (and
  written in languages other than C) to read the representation
  generated.
\end{requirement}

\begin{rationale}
  The simplest way to represent the information GCC needs would be to
  serialize the TREE data structures to disk.  Since these are the
  data structures GCC uses for optimization and code generation, we
  know that these data structures contain the information required.
  However, there are a number of disadvantages to this approach.

  First, different versions of the compiler will almost certainly be
  unable to interoperate.  Minor differences in the tree structure
  from version-to-version of the compiler are inevitable.  In fact,
  different builds of the same version of the compiler may not have
  precisely compatible tree structures; for example, for some
  host/target combinations, either 32-bit or 64-bit
  \code{HOST\_WIDE\_INT}s may be used, but the resulting in-memory tree
  representations are different.  It is highly desirable to support
  link-time optimization of relocatable files built by one provider on
  the systems of another provider, and there is no guarantee that both
  environments are necessarily using the same version of the compiler.

  Second, there are a large number of additional tools that would
  benefit from being able to access the information possessed by the
  compiler in the middle of its processing.  Lint-like source analysis
  tools, IDEs, and other similar tools are all potential clients.
  Various projects to generate XML from GCC have been attempted in
  order to try to facilitate this kind of usage.
\end{rationale}

It might be possible to use a well-known representation format, such
as Java bytecode or Microsoft's CIL.  However, in addition to concerns
about possible intellectual property issues and alignment with
particular vendors, we also concluded that neither of these formats
was well-suited to GCC's needs.  Java bytecode is too Java-specific
for a multi-language compiler.  CIL is too closely tied to Microsoft's
virtual machine architecture.  Therefore, we concluded that it is
necessary to develop a new representation format.

\begin{requirement}
  The representation format should be extensible to contexts other
  than link-time optimization.
\end{requirement}

\begin{rationale}
  The primary purpose of the representation format is for link-time
  optimization.  However, while the link-time optimizer may want to
  process a language-neutral representation of the program that has
  been partially optimized before emission, other tools (such as the
  IDEs and source-analysis tools mentioned above) will likely want to
  access an unoptimized representation of the program that contains
  more information about the way in which the user originally
  formulated the program.  In particular, an implementation of the C++
  ``export'' keyword will want a representation of the program that is
  C++-specific.  GCC's TREE representation is sufficiently flexible to
  represent these various levels; the on-disk representation should be
  similarly flexible.
\end{rationale}

\begin{requirement}
  The semantics of the representation (for each supported level of
  representation) must be well-specified.
\end{requirement}

\begin{rationale}
  In order to facilitate the interoperability described in
  Requirement~\ref{req:interop}, we must have a well-specified
  explanation of the meaning of the representation.  Ideally, we would
  have an operational semantics capable of formally describing
  execution of the input program, but that goal is not realistically
  achievable; instead, we will have to settle for an informal document
  describing the semantics.

  Each level of representation should be documented, as part of the
  GCC manual, and handled through the normal GCC development process.
  We do not intend to encourage ad-hoc extensions to the format for
  the use of particular individual needs.
\end{rationale}

\begin{requirement}
  The representation used for link-time optimization should correspond
  approximately to the GIMPLE TREE representation.
\end{requirement}

\begin{rationale}
  The RTL level would be too low-level a representation for
  link-time optimization as its use would preclude the use of the
  increasingly powerful TREE optimizers.

  GIMPLE is a language-independent representation that has been
  simplified to make it amenable to optimization.  However, high-level
  information, such as the types of expressions remains.

  Before emitting information for link-time optimization, it may be
  useful to apply existing optimizations that reduce program size,
  such as constant propagation, dead code elimination, and, perhaps, a
  limited form of inlining.  Functions with internal linkage that are
  never referenced can be eliminated.

  We do not yet have an opinion as to whether the representation
  should be in SSA form.  While the information provided by SSA form
  is quite useful, it can be easily regenerated.  PHI nodes, and the
  use of additional variables, may so bloat the representation that it
  is more efficient to emit a non-SSA form.

  The use of general compression algorithms may be undesirable, to the
  extent that they make decoding the data difficult, including
  preventing random access to the data.  However, it might still be
  desirable to use such an algorithm, if the compression achieved is
  sufficiently significant.  In addition, some forms of
  context-sensitive compression may be useful.  Almost all
  externalized byte code representations use a representation that is
  a variant of three-address code.  These representations often
  require a large number of intermediate variables.  Schemes, such as
  dividing the program into blocks such that a limited number of
  variables are present in each block, and then using small integers
  to refer to the registers with a given block, can reduce the cost.
\end{rationale}

\begin{requirement}
  The representation format must be designed to support forward
  evolution.  For example, the format should permit the addition of
  new operators as such operators are added to GIMPLE.  It should be
  possible to perform link-time optimization using input object files
  generated with different versions of the compiler, provided that the
  link-time optimizer used is that corresponding to the newest
  compiler used to build the object files.
\end{requirement}

\begin{note}
  This requirement is not meant to imply that an old version of the
  link-time optimizer should be able to process objects created by a
  newer version of the compiler.  Rather, a new version of the
  link-time optimizer should be able to process objects created by an
  older version of the compiler.  There may, occasionally, be cases
  where that is not possible, and the representation format should
  contain version numbers that permit the link-time optimizer to
  identify such cases.
\end{note}

\begin{rationale}
  Since the format is intended for use not only by the link-time
  optimizer, but also by other tools, the format must be fundamentally
  stable.  However, as new optimizations and new language features are
  added to GCC, it may occasionally be necessary to add new operations
  to GIMPLE, new kinds of types, and so forth.  Therefore, the format
  must support forward evolution.
\end{rationale}

\subsection{Declarative Representation}

This subsection outlines the representation of declarative entities,
i.e., types, variables, functions (their declarations, not their
bodies), and other similar entities.  The bodies of functions are
discussed in Section~\ref{sec:execrep} below.

The representation for types will be DWARF-3.  Although the DWARF-3
standard was originally designed for providing debugging information,
it's representation for types provides nearly all of the information
required for link-time optimization, including the location of
non-static data members and base classes.  Furthermore, DWARF-3
specifically provides for vendor extension, allowing us to record any
additional information that we might require.  In particular, we will
extend DWARF-3 to represent relevant GNU attributes.  

\begin{note}
Description of GNU attributes in DWARF-3 will also be of use to GDB
and other debuggers, to the extent that, for example, these attributes
affect the calling conventions of functions using these types.  While
this proposal contemplates the use of the DWARF-3 extensions in
service of link-time optimization, there is no reason the same
extensions should not be used as part of the conventional debugging
information.
\end{note}

DWARF-3 has several other advantages for our purposes, including:
\begin{itemize}
\item DWARF-3 is a well-specified standard.

\item GCC already knows how to emit DWARF-3 information.  Therefore,
  the effort required to emit the information will be much smaller.

\item Various tools, including \code{readelf}, know how to interpret
  and display DWARF-3 information.  Therefore, it will be possible to
  more easily verify the generated information, and construction of the
  portion of the link-time optimizer that will read this information
  and reconstitute appropriate TREE nodes will be easier.
\end{itemize}

The representation for declarations will also be DWARF-3.  For
function declarations and for variables with static storage duration
(i.e., ``global'' variables) this representation is very natural.  For
automatic variables, the representation is less natural, in that
DWARF-3 scope information refers to byte offsets within code sections.
In other words, the DWARF-3 representation for a local variable says
that it comes into scope at offset $N_0$ (from the start of a
designated section) and goes out of scope at offset $N_1$.  References
into the ordinary code sections clearly will not be useful to the
link-time optimizer.  Therefore, a separate DWARF-3 section will be
used which references not the usual code section, but, rather, the
section used by the executable representation.

Some variables with static storage duration require initialization.
If the initialization should be performed dynamically, no special
handling is required; the dynamic initialization will appear in the
ordinary code stream.  If the initialization is performed statically,
the initializer will be recovered directly from the data section in
the relocatable file, as there is no convenient representation for
initialization in DWARF-3.

\subsection{Executable Representation}
\label{sec:execrep}

The executable portion of the intermediate code will be ``bytecode''
for a stack machine hereinafter known as the GNU Virtual Machine
(GVM).  The stack machine will represent code at approximately the
same level represented by the GIMPLE level of TREE.  The use of a
stack removes any need for pickled pointers to represent expression
trees.  The TREE representation of the executable code can be readily
reconstituted from the stack code.

The GVM will share some features with Sun's Java Virtual Machine (JVM)
and Microsoft's Common Intermediate Language (CIL).  However, there
will also be some important differences:
\begin{description}

\item[portability] Like JVM bytecode, the format of the GVM bytecode
  will be architecture-independent.  However, unlike JVM bytecode, it
  will not be possible to reuse the GVM bytecode from one platform on
  another; for example, the bytecode generated by an IA32 GNU/Linux
  will not be reusable as part of a PowerPC AIX program.
  
\item[executability] Given that the GVM code will be translated into
  executable code, it is highly likely that this code could be
  directly executed.  However, no particular attention will be given
  to make this easy.  What is important is to be able to transport the
  intermediate code efficiently from the compiler to the link time
  optimizer.

\item[verification] Some bytecode formats permit verification; i.e., a
  tool can check that executing the bytecode will not result in
  certain kinds of unsafe behaviors.  The Java bytecode format is
  fully verifiable.  The CIL format provides optional verification.
  The GVM format will not be designed to support verification because
  most of our target languages explicitly permit operations that would
  be considered unsafe by a verifier.

\item[operators] The operators will be drawn from two sets: those
  that manipulate the stack ({\code load, store, duplicate, swap)} and
  the operators used in GIMPLE.

  \begin{note} 
  We believe that it may be useful to add some higher-level operations
  to GIMPLE, such as array operations, virtual functions calls,
  dynamic casts, parallel execution blocks, and high level loops.  To
  the extent that these higher-level operators will facilitate
  additional optimizations, they should be present in GIMPLE, and,
  therefore, in the stack machine.  As GIMPLE evolves, the stack
  machine should evolve in parallel.
  \end{note}

\item[the stack] The stack is an unbounded array of expressions.  It
  will be an invariant of the code generated that the stack will be
  empty at basic block boundaries.  Basic block boundaries will be
  explicitly indicated in the stack machine code.

  JVM and CIL codes utilize a register allocation pass to generate
  their intermediate code. This register allocation obscures the code
  in a way that makes debugging more difficult and requires
  significant resources on both the input and output sides.  Therefore,
  the generation of GVM bytecode will not require register allocation.

\item[blocks] There will be several types of explicit blocks in the
  code stream. These are basic blocks, bind expressions, parallel
  execution blocks.  

\item[the registers] The registers of the GVM correspond one for one
  with the local variables at the GIMPLE level.  Each local variable
  will be assigned an index, and a symbol table will be provided to
  keep the type of each of these temps.  Types will be given as 
  references to the corresponding DWARF-3 DIEs representing those
  types.

\item[types] The stack is dynamically typed.  The type of an element on the
  stack is determined from the operation used to set that item.  The
  result type of a generic operation (e.g., addition) is determined
  from the inputs to that operation.  Some operations (e.g., casts)
  will explicitly indicate the result of the operation.

\end{description}



\end{document}
